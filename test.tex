\documentclass[12pt, oneside]{article}
\input{./CS40packages.tex}
\usepackage{enumitem}
\title{HW3 Collaborative}
\author{CS40 Winter '24}
\date{Due: Wednesday, February 07, 2024 at 11:59PM on Gradescope}
\begin{document}
\maketitle
{\bf For all Collaborative HW assignments:}
Collaborative homeworks may be done individually or in groups of
up to 4 students. You may switch HW partners for different HW
assignments. Please ensure your name(s) and PID(s) are clearly
visible on the first page of your homework submission.
All submitted homework for this class must be typed. Diagrams may
be hand-drawn and scanned and included in the typed document. You
can use a word processing editor if you like (Microsoft Word,
Open Office, Notepad, Vim, Google Docs, etc.) but you might find it
useful to take this opportunity to learn LaTeX. LaTeX is a markup
language used widely in computer science and mathematics. The
homework assignments are typed using LaTeX and you can use the
source files as templates for typesetting your solutions.\
footnote{To use this template, you will need to copy both the
source file (extension \texttt{.tex}) you'll be editing
and the file containing all the ``shortcut" commands we've defined
for this class
\href{https://drive.google.com/file/d/1FmQvgByKnNjTpIkAUw31TGWYrQZ
M-HK0/view?usp=sharing})}
{\bf Integrity reminders}
\begin{itemize}
\item Problems should be solved together, not divided up between
the partners. The homework is
designed to give you practice with the main concepts and
techniques of the course, while getting to know and learn from
your classmates.
\item You may not collaborate on homework with anyone other than
your group members.
You may ask questions about the homework in office hours (of the
instructor, TAs, and/or tutors) and
on Piazza. You \emph{cannot} use any online resources about the
course content other than the text
book and class material from this quarter -- this is primarily to
ensure that we all use consistent notation and
definitions we will use this quarter.
\item Do not share written solutions or partial solutions for
homework with other students in the class who are not in your
group. Doing so would dilute their learning experience and
detract from their success in the class.
\end{itemize}
You will submit this assignment via Gradescope
(\href{https://www.gradescope.com}{https://www.gradescope.com})
in the assignment called ``HW3-Collaborative''.
\subsection*{Summary of Proof Strategies (so far)}
In your proofs and disproofs of statements below, justify each
step
by reference to a component of the following proof strategies
we have discussed so far, and/or to relevant definitions and
calculations.
\begin{itemize}
\item A counterexample can be used to prove that $\forall x
P(x)$ is {\bf false}.
\item A witness can be used to prove that $\exists x P(x)$
is {\bf true}.
\item {\bf Proof of universal by exhaustion}: To prove that $\
forall x \, P(x)$
is true when $P$ has a finite domain, evaluate the predicate at {\
bf each} domain element to confirm that it is always T.
\item {\bf Proof by universal generalization}: To prove that
$\forall x \, P(x)$
is true, we can take an arbitrary element $e$ from the domain and
show that $P(e)$ is true, without making any assumptions about
$e$ other than that it comes from the domain.
\item To prove that $\exists x P(x)$ is {\bf false}, write
the universal statement that is logically equivalent to its
negation and then prove it true using universal generalization.
\item {\bf Proof of Conditional by Direct Proof}: To prove
that the implication $p \to q$ is true, we can assume $p$ is true
and use that assumption to show $q$ is true.
\end{itemize}
\newpage
\section*{Assigned Questions}
\begin{enumerate}
\item For each of these arguments, identify what rule of
inference is used.\footnote{See zyBooks 3.5 for list of rules of
inference}
\begin{enumerate}
\item Kangaroos live in Australia and are marsupials.
Therefore, kangaroos are marsupials. (Simplification)
\item It is either hotter than 100 degrees today or the
pollution is dangerous. It is less than 100 degrees outside
today. Therefore, the pollution is dangerous. (Disjunctive syllogism)
\item Linda is an excellent swimmer. If Linda is an excellent
swimmer, then she can work as a lifeguard. Therefore, Linda can
work as a lifeguard. (Modus Ponens)
\item If I work all night on this homework, then I can answer
all the exercises. If I answer all the exercises, I will
understand the material. Therefore, if I work all night on this
homework, then I will understand the material. (Hypothetical syllogism)
\end{enumerate}
\item For each of these arguments, determine whether the argument
is correct or incorrect and explain why. If the argument is
correct, state which rule of inference is used, and if it is
incorrect, give a reasonable explanation for why it is incorrect.
\begin{enumerate}
\item Everyone enrolled in the university has lived in a
dormitory. Mia has never lived in a dormitory. Therefore, Mia is
not enrolled in the university. (Correct: Modus Tollens)
\item A convertible car is fun to drive. Isaac’s car is not a convertible. Therefore, Isaac’s car is not
fun to drive. (Incorrect: the premise does not mean that convertibles are the only cars that are fun to drive)
\item Quincy likes all action movies. Quincy likes the movie
{\it Eight Men Out}. Therefore, Eight Men Out is an action movie.
(Incorrect: the premise does not contradict the fact that Quincy may like other genres. {\it Eight Men Out} may be one of the other genres she likes)
\item They allow you into the dining commons only if you have
a green badge. Olivia is inside the dining commons. Therefore,
Olivia has a green badge.
(Correct: Modus Ponens)
\end{enumerate}
\item Is the following argument valid or invalid? Prove your
answer by replacing each proposition with a variable to obtain
the form of the argument. Then prove that the form is valid or
invalid using truth tables.
I will buy a new stereo system and new sunglasses only if I get a
promotion.\\
I am not going to get promoted.\\
I will buy new sunglasses.\\
Therefore, I will not buy a new stereo system. \\
\newline Using notation:
\begin{enumerate}
    \item $ST$ - I buy a stereo system
    \item $SG$ - I buy sunglasses
    \item $P$ - I am promoted
\end{enumerate}
\vspace{-1em}
\newpage
The argument can be re-written as:
\newline $(ST \land SG) \rightarrow P$ (Premise 1)
\newline $\neg P$ (Premise 2) 
\newline $SG$ (Premise 3)
\newline $\rule{2cm}{0.1mm}$
\newline $\therefore \neg ST$

Proof by truth table (0 for False, 1 for True):
\begin{table}[htbp]
    \centering
    \begin{tabular}{|c|c|c|c|c|c|}
    \hline 
        \textbf{ST} & \textbf{SG} & \textbf{P} & $\neg P$ & $(ST \land SG) \to P$ & $\neg ST$ \\
    \hline
        0 & 0 & 0 & 1 & 1 & 1\\
    \hline
        0 & 0 & 1 & 0 & 1 & 1\\
    \hline
        0 & 1 & 0 & 1 & 1 & 1\\ 
    \hline
        0 & 1 & 1 & 0 & 1 & 1\\ 
    \hline
        1 & 0 & 0 & 1 & 1 & 0\\ 
    \hline
        1 & 0 & 1 & 0 & 1 & 0\\
    \hline
        1 & 1 & 0 & 1 & 0 & 0\\ 
    \hline
        1 & 1 & 1 & 0 & 1 & 0\\ 
    \hline
    \end{tabular}
\end{table}

The only row that  satisfies all three premises is Row 3. The conclusion is True for Row 3. Hence, this is a \textbf{valid} argument.

\item The domain for variables $x$ and $y$ is a group of people.
The predicate $F(x, y)$ is true if and only if $x$ is a friend of
$y$. For the purposes of this problem, assume that for any person
$x$ and person $y$, either $x$ is a friend of $y$ or $x$ is an
enemy of $y$. Therefore, $\neg F(x, y)$ means that $x$ is an
enemy of $y$.
Translate each statement into a logical expression. Then negate
the expression by adding a negation operation to the beginning of
the expression. Apply De Morgan's law until the negation
operation applies directly to the predicate and then translate
the logical expression back into English.
% \vpsace{-2em}
\begin{enumerate}[noitemsep]
\item Everyone has an enemy. $\forall x \exists y \neg F(x, y)$
\item Everyone is their own friend. $\forall x F(x, x)$
\item At least two different people are friends. $\exists x \exists y ((x \neq y) \land F(x, y))$
\item``The enemy of my enemy is my friend" \footnote{For all
people in the domain, enemies of their enemies are their
friends}.
$\forall x \forall y \forall z ( ((\neg F(x, y) \land \neg F(y, z)) \to F(x, z) )$ (PS: the domain for z is also a group of people)
\end{enumerate}
\item Assuming that the domains of all quantifiers are the same,
use rules of inference to show that
$$ \text{if } \forall x(P(x)\lor Q(x)) \land \forall x((\neg
P(x)\land Q(x)) \to R(x)) \text{, }$$
$$\text{then } \forall x(\neg R(x) \to P(x)) . \\ $$

\begin{enumerate}
    \item $\forall x (P(x) \lor Q(x))$ (Hypothesis)
    \item $\forall x (~(\neg P(x)\land Q(x)) \to R(x)~)$ (Hypothesis)
    \item Let c be an arbitrary constant (Variable definition)
    \item $P(c) \lor Q(c)$ (Universal instantiation, a)
    % \item $(\neg P(c) \land Q(c)) \to R(c)$ (Universal Instantiation, b)
    % \item $(P(c) \lor \neg Q(c)) \lor R(c)$ (Implication Law, d)
    % \item $P(c) \lor \neg Q(c) \lor R(c)$ (Associativity Law, e)
    \item $(~(\neg P(c)\land Q(c)) \to R(c)~)$ (Universal Instantiation, b)
    \item $\neg (\neg P(c)\land Q(c)) \lor R(c)$ (Implication Law, e) 
    \item $\neg \neg (P(c) \lor \neg Q(c)) \lor R(c)$ (De Morgan's Law, f)
    \item $(P(c) \lor \neg Q(c)) \lor R(c)$ (Double Negation Law, g)
    \item $P(c) \lor \neg Q(c) \lor R(c)$ (Associativity Law, h)
    \item Assume $\neg R(x)$
    \item $(P(c) \lor \neg Q(c))$ (Disjunctive Syllogism, i, j)
    \item Proof that P(c) is True (k)
    \item Case 1: $P(c)$ is True
    \item Case 2: $\neg Q(c)$ is True
    \item If Case 2, P(c) must be True (Disjunctive Syllogysm, d, n)
    \item $\therefore \neg R(c) \to P(c)$ (l)
    \item $\therefore \forall x (\neg R(x) \to P(x))$ (Universal Generalization, p)
\end{enumerate}

\item Consider the predicate $F(a,b) = ``a \text{ is a factor
of } b"$ over the domain $\mathbb{Z}^{\neq 0} \times \mathbb{Z}$
that was introduced in lecture. Consider the following quantified
statements
\label{factoring}
\begin{multicols}{2}
\begin{enumerate}[label=(\roman*)]
\item $\forall x \in \mathbb{Z} ~(F(1,x))$
\item $\forall x \in \mathbb{Z}^{\neq 0} ~(F(x,1))$
\item $\exists x \in \mathbb{Z} ~(F(1,x))$
\item $\forall x \in \mathbb{Z}^{\ne 0} ( \neg F(x,1))$
\item $\forall x \in \mathbb{Z}^{\neq 0} ~\exists y \in \
mathbb{Z} ~(F(x,y))$
\item $\exists x \in \mathbb{Z}^{\neq 0} ~\forall y \in \
mathbb{Z} ~(F(x,y))$
\item $\forall y \in \mathbb{Z} ~\exists x \in \mathbb{Z}^{\neq
0} ~(F(x,y))$
\item $\exists y \in \mathbb{Z} ~\forall x \in \mathbb{Z}^{\neq
0} ~(F(x,y))$
\end{enumerate}
\end{multicols}
\begin{enumerate}
\item ({\it Graded for correctness of choice and fair effort
completeness in justification})
Which of the statements (i) - (viii) is being {\bf proved} by the
following proof?
\begin{quote}
By universal generalization, {\bf choose} $e$ to be an {\bf
arbitrary} integer.
We need to show $\exists y \in \mathbb{Z}^{\neq 0} (F(y,e))$. By
definition of the predicate $F$, we can rewrite
this goal as $\exists y \in \mathbb{Z}^{\neq 0} \exists c \in
\mathbb{Z}~(e = c \cdot y)$. We pick the {\bf witnesses} $y = 1$
and $c = e$. $y$ is a non-zero integer and therefore in the
domain. Similarly, $c$ is an integer and therefore in the domain.
Plugging the value of the witnesses $y$ and $c$, we get
$c \cdot y = e \cdot 1 = e$, as required. Since the predicate $\
exists y \in \mathbb{Z}^{\neq 0} (F(y,e))$ evaluates to true
for the arbitrary integer $e$, the claim has been proved.
\hfill $\blacksquare$
\end{quote}
{\it Hint: It may be useful to
identify the keywords in the proof that indicate proof
strategies.}
\newline \textbf{(vii)}
\item ({\it Graded for correctness of choice and fair effort
completeness in justification})
Which of the statements (i) - (viii) is being {\bf disproved} by
the following proof?
\begin{quote}
To disprove the statement, we need to find a counterexample. We
choose $-1$, which is a nonzero
integer so in the domain. We need to show $ F(-1,1)$. By
definition of the predicate $F$, we
can rewrite this goal as $\exists c \in \mathbb{Z}~(1 = c \cdot
-1)$. We pick the {\bf witness} $c = -1$,
which is an integer and therefore in the domain. Plugging the
value of the witness $c$, we get
$c \cdot -1 = -1 \cdot -1 = 1$, as required. So the
counterexample works to
disprove the original statement.
\hfill $\blacksquare$
\end{quote}
{\it Hint: It may be useful to
identify the key words in the proof that indicate proof
strategies.}
\newline \textbf{(iv)}
\end{enumerate}
\item\label{predicate} ({\it Graded for correctness of evaluation
of statement (is it true or false?) and fair effort completeness
of the translation and proof}) Translate the following statement
to English and then prove or disprove it:
$$\forall x \in \mathbb{Z}^{\neq 0}~ \forall y \in \mathbb{Z}^{\neq 0} ~(~F(x,y) \to F(x,x+y)~)$$

For every two nonzero integers $x$ and $y$, if \textit{x} divides \textit{y}, then \textit{x} also divides $x+y$

Proof:

Let c and k be arbitrary nonzero integers. Then, we need to prove that 
$F(c, k) \to F(c, c+k)$ evaluates to True. To do that, we need to show that when F(c, k) is True, it must also be that F(c, c+k) is also True.

By definition of F: $F(c, k) \equiv k = mc$ for some integer m. For the sake of this proof, we will assume that it is True.

Then, add c to both sides of the $k = mc$ equation. This yields the equation $k + c = mc + c$. By laws of algebra, this equation is equivalent to $k + c = c(m + 1)$. Since m is an integer, m + 1 is also an integer. Let n = m + 1 (n is also an integer because m + 1 is an integer). Plugging n into the previous equation yield $k + c = cn$ (1). (1) is True due to the assumption that F(c, k) is True.

By definition of F: $F(c, k + c) \equiv k + c = cv$ for some integer v (2).

Plugging (2) into (1) yields $cv = cn$. Dividing both sides by c gives us $v = n$. Since v and n are equal, we can plug n into (2): $k + c = cv \equiv k + c = cn$. Because the RHS is (1), which is True, we can conclude that $k + c = cv$ is also True. By (2), $F(c, k + c) \equiv k + c = cv \equiv True$.

Therefore, if F(c, k) is True, then F(c, c + k) is True as well.
\hfill $\blacksquare$

\pagebreak
\item Let $P = \{a: a\in \bbscript{Z}^+, a \leq 50 \}$. We want
to list all pairs $(a, b)$ of integers in $P\times P$ such that
$a$ is a factor of $b$ (which is the same as the predicate $F(a,
b)$ that you have been working in the previous question).
The main constraint in this questions is that we want to build
this list without directly listing all pairs (because it would be
time consuming) and without ever evaluating the predicate $F(a,
b)$ because it would involve performing a modulo operation. So,
we have decided to start with an arbitrarily created initial set
$K \subseteq P \times P$, defined as $K = \{(1,2), (1, 4), (2,4),
(3,9)\}$, knowing well that for all pairs of numbers in $K$, the
first number is a factor of the second but that $K$ does not
contain all pairs of interest. Our goal is then to recursively
build up the set $L \subseteq P \times P$ by:
\begin{itemize}
\item using some assumptions that we know to be true about
integer pairs that are related by the property of one being the
factor of the other, and
\item only checking for membership of elements in $K$, $P$,
and $L$ but never directly evaluating the predicate $F(a, b)$.
\end{itemize}
As a starting guess, we decide to use the following
assumptions:\\
{\em (i)} for all $a$, $b \in P$, if $a$ is a factor of $b$ then
$a$ is a factor of $ab$,\\ {\em (ii)} for all $a$, $b$, $c$ in
$P$ such that $a \neq c$, $a\neq b$ and $b \neq c$, if $a$ is a
factor of $b$ and $b$ is a factor of $c$ then $a$ is a factor of
$c$, and \\{\em (iii)} for all $a$, $b$, $c$ in $P$ such that
$a \neq c$, $a\neq b$ and $b \neq c$, if $a$ is a factor of $b$
and $a$ is a factor of $c$ then $a$ is a factor of $b+c$\\
\begin{enumerate}
\item\label{rec_set} Write a recursive definition for a set $L \subseteq P \times P$ that captures the following notion: $(a,
b) \in L$ if and only if we can deduce from $K$ and our
assumptions $(i)$, $(ii)$, $(iii)$ that $a$ is a factor of $b$.
Once again, in writing this definition, you are only allowed to
check for membership of elements in the sets $K$, $P$, and $L$
but never directly evaluating the predicate $F(a, b)$

PS: it is possible to write this in one line. Separating into multiple lines for the ease of reviewing.
$L_0 = \{(a, b) \in K\}$ (Base Case)

$L_1 = L_0 \cup \{(a, ab) \mid (a, b) \in L \land ab <= 50\}$ (i)

$L_2 = L_1 \cup \{(a, c) \mid (a, b) \in L \land (b, c) \in L \land (a \neq c) \land (a \neq b) \land (b \neq c)\}$ (ii)

$L_3 = L_2 \cup \{(a, b + c) \mid (a, b) \in L \land (a, c) \in L \land (a \neq c) \land (a \neq b) \land (b \neq c) \land (a, b+c) \in P \times P\}$ (iii)

$L = L_3$

\item List five example elements in the set $\{(a,b) \in P\times P
\mid F(a,b)\}$ that are each not in the set $L$ as per your
recursive definition in~(\ref{rec_set}). Write your answer in
roster notation.
\newline $\{(8, 16), (9, 18), (10, 20), (11, 22), (12, 24)\}$

\item Modify the recursive definition from~(\ref{rec_set}) so that
that the resulting set contains all pairs of numbers in $P \times P$, where the first number is a factor of the second number.
The same constraints as part~(\ref{rec_set}) apply here with the
exception that in this question you may use a different set of
assumptions than the ones provided as a starting guess. Write
your solution in a way so that it generalizes well to any
different choice of the sets $P$ and $K$. Note that although non-
recursive definitions are possible, you must provide a recursive
definition to receive credit for this question.

Let $a_0 be the lowest possible number in P$

$L_0 = \{(a, b) \in K\} \cup \{(a_0, a_0)\}$ (Base Case with smallest element included)

$L_1 = L_0 \cup \{(a, ab) \mid (a, b) \in L \land B(ab)\}$ (i)

$L_2 = L_1 \cup \{(a, c) \mid (a, b) \in L \land (b, c) \in L \land (a \neq c) \land (a \neq b) \land (b \neq c)\}$ (ii)

$L_3 = L_2 \cup \{(a, b + c) \mid (a, b) \in L \land (a, c) \in L \land (a \neq c) \land (a \neq b) \land (b \neq c) \land B(b+c)\}$ (iii)

% b = ma, b + a = ma + a => b + a = m(a + 1)
$L_4 = L_3 \cup \{(a + 1, b + a) \mid (a, b) \in L \land (a+1) \in P \land (b + a) \in P\}$

$L = L_4$

\item Critique your solution for part(c) by discussing whether it
generalizes to any choice of the sets $P \subseteq \mathbb{Z^+} $
and $K \subseteq P \times P$, and if so under what assumptions.

Assuming that K is non-empty, I believe that this solution generalizes to any choice of sets. That is, because solution ~(\ref{rec_set}) had two issues. 

First, if K did not have the smallest element of P, then it cannot include elements with numbers lower than given initially. Base step solves this case by introducing minimum element ($(a_0, a_0)$)

Second, the solution in ~(\ref{rec_set}) could only increment the second number of a pair (i.e. not \textit{a} in \textit{(a, b)}). The solution in part c solves that with rule in $L_4$.

\end{enumerate}
\end{enumerate}
\vfill
\subsection*{Attributions}
Some of the problems on this homework are based on questions
originally created by Mia Minnes, Joe Politz, and Daniel
Lokshtanov.
\end{document}